% ----------------------------------------------------------------
% Article Class (This is a LaTeX2e document)  ********************
% ----------------------------------------------------------------
\documentclass{article}
\usepackage[english]{babel}
\usepackage{styfiles/proof, styfiles/code, amsmath,amsthm}
\usepackage{bbold}%for typeface: mathbb
\usepackage{hyperref}
% THEOREMS -------------------------------------------------------
\newtheorem{thm}{Theorem}[section]
\newtheorem{cor}[thm]{Corollary}
\newtheorem{lem}[thm]{Lemma}
\newtheorem{prop}[thm]{Proposition}
\theoremstyle{definition}
\newtheorem{defn}[thm]{Definition}
\theoremstyle{remark}
\newtheorem{rem}[thm]{Remark}
\numberwithin{equation}{section}

% ----------------------------------------------------------------
\begin{document}

\newcommand{\env}[1]{[\![#1]\!]\kappa}
\newcommand{\round}[1]{(\!|#1|\!)}

\title{CPS Conversion}%
\author{Di Zhao}%
%\address{address}%
%\thanks{}%\sqrt{}
\date{\small{\texttt{zhaodi01@mail.ustc.edu.cn}}}%
% ----------------------------------------------------------------

\maketitle
% ----------------------------------------------------------------

Document related to the CPS conversion phase.

\section{ML Syntax}

\begin{figure}[!ht]
  % Requires \usepackage{graphicx}
  \centering
\begin{tabular}{rclr}
$e$ & $\to$ & $x$ & \emph{variable}\\
  %   &   & $t$ \textsf{primop} $t'$& \emph{arithmetic term}\\
     & $|$   & \texttt{()} & \emph{empty}\\
     & $|$   & \texttt{true} & \emph{true}\\
     & $|$   & \texttt{false} & \emph{false}\\
     & $|$   & $i$ & \emph{integer}\\
     & $|$   & $"s"$ & \emph{string}\\
     &  $|$  & \texttt{fn} $x$\texttt{ => }$e$ & \emph{abstraction}\\
     &  $|$  & $e_1\ e_2$ & \emph{application}\\
     &  $|$  & ($e_1,e_2, ..., e_n$) & \emph{tuple}\\
     &  $|$  & \texttt{\#}$i\ e$  & \emph{projection}\\
     &  $|$  & \textsf{in}$i\ e$  & \emph{tagged value}\\
     &  $|$  & \textsf{case }$e$ \textsf{of }
            $\overrightarrow{\texttt{in}i_j\ x_j\texttt{ => } e_j}$ & \emph{case}\\
     &  $|$  & $PrimOp\ \vec{e}$ & \emph{operations}\\
     &  $|$  & \textsf{let val }$x=e_1$\textsf{ in }$e_2$\textsf{ end}
         & \emph{let}\\
     &  $|$  & \textsf{if }$e$\textsf{ then }$e_1$\textsf{ else }$e_2$
         & \emph{if}\\
     &  $|$  & \textsf{let fix }$f\ x=e_1$\textsf{ in }$e_2$\textsf{ end}
         & \emph{fix}\\
$PrimOp$ & $\to$ & $+$ & \emph{add}\\
     & $|$ & $-$ & \emph{sub}\\
     & $|$ & $*$ & \emph{times}\\
     & $|$ & $>$ & \emph{less than}\\
     & $|$ & $<$ & \emph{larger than}\\
     & $|$ & $=$ & \emph{equals}\\
     & $|$ & $\texttt{not}$ & \emph{not}\\
     & $|$ & $\texttt{andalso}$ & \emph{and also}\\
     & $|$ & $\texttt{orelse}$ & \emph{or else}\\
     & $|$ & $\texttt{print}$ & \emph{print}\\
     & $|$ & $\texttt{int2string}$ & \emph{toString}\\

\end{tabular}
  \caption{ML Syntax}
  \label{fig-sub}
\end{figure}

Figure 1 illustrates the syntax of our source language - a subset
 of ML. Here we use the metavariable $e$ to represent an arbitrary
 expression of the source language. Similarly, $x$ is a metavariable
ranging over variables.\\

\noindent{Updates:
\begin{description}
  \item[15-7-1:]The case expression is changed to: \textsf{case }$e$ \textsf{of }
            $\overrightarrow{\texttt{in}i_j\ x_j\texttt{ => } e_j}$ ,
to expand the original dualistic cases into indefinite cases, to facilitate generating the
 \texttt{apply} function in the defunctionalization phase. $i$ is the integer representing
 the type constructor. We need a front end
to map the constructors in \texttt{datatype} definitions with integers.

We may \textcolor[rgb]{1.00,0.00,0.00}{\textbf{need a typing system}}
to generated executive ML code for each
intermediate representation. (We don't know how many labels there are.)

  \item[15-9-5:]
  Added boolean values and corresponding operations. Changed \texttt{if0}
  expression into \texttt{if} expression.
\end{description}
}

\section{CPS Syntax}

Figure 2 illustrates the syntax of the CPS language corresponding
to the ML syntax in Figure 1. In the CPS syntax, we introduce the metavariable
 $k$ to represent a continuation. \\

\begin{figure}[!ht]
  % Requires \usepackage{graphicx}
  \centering
\begin{tabular}{rrcl}
(terms) & $K$ & $\to$ & \textsf{letval }$x\ =\ V$ \textsf{ in } $K$ \\
        &     & $|$ & \textsf{letcont }$k\ x = K$\textsf{ in }$K'$\\
        &     & $|$ &  $k\ x$ \\
        &     & $|$ & $f\ k\ x$ \\
        &     & $|$ & \textsf{case} $x$ \textsf{of}
            $\overrightarrow{\texttt{in}i_j\ x_j\ \texttt{=>}\ K_j}$\\
        &     & $|$ & \textsf{letprim} $x=PrimOp\ \vec{y}
         \texttt{ in}\ K$\\
        &     & $|$ &\textsf{if} $x$ \textsf{then} $k_1\ \textsf{else}\ k_2$\\
        &     & $|$ &\textsf{letfix }$f\ k\ x=K$\textsf{ in }$K'$\\

(values) & $V$ & $\to$ & () $\ \ |$ \textsf{ true } $|$ \textsf{ false}\\
        &     & $|$ & $i\ \ $ $\ |\ $  $\ "s"$\\
        &     & $|$ & ($x_1,x_2, ..., x_n$)\\
        &     & $|$ & \textsf{in}$i\ x$\\
        &     & $|$ &  $\lambda k\ x.K$ \\
        &     & $|$ &  $\texttt{\#}i\ x$ \\

(primitive & $PrimOp$ & $\to$ & $+\ $ $|\ -\ |$ $\ *$\\
operations) &     & $|$ & $>\ |\ <\ |\ =$\\
        &     & $|$ & $\textsf{andalso }\ |\ \textsf{orelse }\ |\ \textsf{ not}$\\
        &     & $|$ & \textsf{print} $|$ \textsf{int2string}\\
\end{tabular}
  \caption{CPS syntax}
  \label{fig-sub}
\end{figure}



\noindent{
Updates:
\begin{description}
  \item[15-7-1:]
  Removed $\texttt{let}\ x = \pi _i\ y\ \texttt{in}\ K$ to simplify the syntax
  and rules. Instead, add $\texttt{\#}i\ x$ to the values.

  Change the form of \texttt{case} to enable multiple cases (and to
  make the transformations look better?).

   \item[15-9-5:]
  Added boolean values and corresponding operations. Changed \texttt{if0}
  expression into \texttt{if} expression.
\end{description}
}


\section{CPS Conversion}

In this section we will discuss how to perform CPS conversion.
Expressions in ML can be translated into untyped CPS
terms using the function shown in Figure 3. This is an adaptation
of the standard higher-order one-pass call-by-value transformation
(Danvy and Filinski 1992).

 \begin{figure}[!ht]
  % Requires \usepackage{graphicx}
  \centering
\begin{align*}
 [\![\cdot]\!] \ &\ :\ \textrm{ML}\rightarrow(\textrm{Var}\rightarrow
    \textrm{CTm})\rightarrow \textrm{CTm}\\     %head
   \env{x} \  &=\  \kappa (x)\\             %var
 \env{\texttt{()}} \  &=\  \texttt{letval}\ x=()\ \texttt{in}\ \kappa (x)\\ %empty
\env{\texttt{true}} \  &=\  \texttt{letval}\ x=\texttt{true}\ \texttt{in}\ \kappa (x)\\ %true
\env{\texttt{false}}\  &=
    \  \texttt{letval}\ x=\texttt{false}\ \texttt{in}\ \kappa (x)\\ %false
 \env{i} \  &=\  \texttt{letval}\ x=i\ \texttt{in}\ \kappa (x)\\    %int
 \env{"s"} \  &=\  \texttt{letval}\ x="s"\ \texttt{in}\ \kappa (x)\\    %string
 \env{e_1\ e_2} \  &=\  [\![e_1]\!](\mathbb{\Lambda}z_1.[\![e_2]\!]
 (\mathbb{\Lambda}z_2.\texttt{letcont}\ k\ x=
    \kappa (x)\ \texttt{in}\ z_1\ k\ z_2))\\ %app
  \env{(e_1, ... , e_n)}
        &=\round{[e_1, ... , e_n], \texttt{nil}}(\mathbb{\Lambda}\vec{l}.\\
        &\kern1cm \texttt{letval}\ x=\texttt{tuple}(\vec{l})\
        \texttt{in}\ \kappa (x))\\            % tuple
   [\![\texttt{in}i\ e]\!]\kappa \  &=\  [\![e]\!](\mathbb{\Lambda}z.
        \texttt{letval}\ x=\texttt{in}i\ z\ \texttt{in}\ \kappa (x))\\    %tag
  [\![\texttt{\#}i\ e]\!]\kappa \  &=\  [\![e]\!](\mathbb{\Lambda}z.
        \texttt{letval}\ x=\texttt{\#}i\ z\ \texttt{in}\ \kappa (x))\\       %proj
  \env{\texttt{fn }x\texttt{ => }e}\  &=\ \texttt{letval}\ f=\lambda k\ x.[\![e]\!]
    (\mathbb{\Lambda}z.k\ z)\ \texttt{in}\ \kappa (f)\\     %abstract
  [\![\texttt{let\ val}\ x=e_1\ \texttt{in}\ e_2\ \texttt{end}]\!]\kappa
   &= \texttt{letcont}\ j\ x=[\![e_2]\!]\ \kappa\ \texttt{in}
            \ [\![e_1]\!](\mathbb{\Lambda}z.j\ z)\\     %letval
  \kern-1.2cm [\![\texttt{case}\ e\ \texttt{of}
  \ \overrightarrow{\texttt{in}i_j\ x_j\ \texttt{=>}\ e_j}]\!]\kappa
     &=\ [\![e]\!](\mathbb{\Lambda}z.\texttt{letcont}\ k_0\ x_0=
            \kappa (x_0)\ \texttt{in}\\
         &\kern1cm   \texttt{letcont}\ k_1\ x_1=
            [\![e_1]\!](\mathbb{\Lambda}z.k_0\ z)\ \texttt{in}\\
     &\kern1.2cm \texttt{letcont}\ k_2\ x_2=[\![e_2]\!]
        (\mathbb{\Lambda}z.k_0\ z)\ \texttt{in}\\
     &\kern1.6cm  ...\\
      &\kern2cm \texttt{letcont}\ k_n\ x_n=[\![e_n]\!]
        (\mathbb{\Lambda}z.k_0\ z)\ \texttt{in}\\
      & \kern2.4cm \texttt{case}\ z\ \texttt{of}
      \ \overrightarrow{\texttt{in}i_j\ y_j\ \texttt{=>}\ k_j\ y_j})\\
      & (\textrm{where}\ j = 1,2,\ ...\ ,n)\\    %case
     \env{PrimOp\ \vec{e}\ }
       &=\round{\vec{e}, \texttt{nil}}(\mathbb{\Lambda}\vec{l}.   \texttt{letprim}
       \ x=PrimOp\ \vec{l}\ \texttt{in}\ \kappa (x)))\\  %prims
   \env{\texttt{let fix }f\ x\texttt{ = }e_1 \texttt{ in }e_2}\  &=
   \ \texttt{letfix}\ f\ k\ x=[\![e_1]\!]
    (\mathbb{\Lambda}z.k\ z)\ \texttt{in}\ \env{e_2}\\     %fix
    \kern-1.2cm [\![\texttt{if}\ e_1\ \texttt{then}\ e_2
        \texttt{ else}\ e_3]\!]\kappa
     &=\ [\![e_1]\!](\mathbb{\Lambda}z.\texttt{letcont}\ k_0\ x_0=
            \kappa (x_0)\ \texttt{in}\\
         &\kern1cm   \texttt{letcont}\ k_1\ x_1=
            [\![e_2]\!](\mathbb{\Lambda}z.k_0\ z)\ \texttt{in}\\
     &\kern1.2cm     \texttt{letcont}\ k_2\ x_2=[\![e_3]\!]
        (\mathbb{\Lambda}z.k_0\ z)\ \texttt{in}\\
     &\kern2cm    \texttt{if}\ z\ \texttt{then}\ k_1\ \texttt{else}\ k_2)    %if0
\end{align*}
  \caption{CPS Conversion}
  \label{fig-sub}
\end{figure}

\begin{figure}[!ht]
  % Requires \usepackage{graphicx}
  \centering
\begin{align*}
    \round{\cdot}\ :\ \textrm{ML list * string list}&\to(\textrm{string list}\to
     \textrm{CTm})\to \textrm{CTm}\\     %head
     \round{\texttt{[]}, \omega}\eta
        &=\eta (\texttt{rev} (\omega))\\   %empty
     \round{e::es, \omega}\eta
       &=\  [\![e]\!](\mathbb{\Lambda}x.
          \round{es,x::\omega}\eta )     %some
\end{align*}

  \caption{CPS Conversion for tuples}
  \label{fig-sub}
\end{figure}

$\ $\\

\noindent{
Updates:
\begin{description}
  \item[15-7-2:] Modified the conversion rule for \texttt{case} to enable
  multiple cases. Modified the rule for projection operation (see updates for 15-7-1).
  
\item[15-9-5:]
  Added conversion rules for boolean values. Changed \texttt{if0}
  expression into \texttt{if} expression.
\end{description}
}


\end{document}
