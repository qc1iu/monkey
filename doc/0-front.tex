% ----------------------------------------------------------------
% Article Class (This is a LaTeX2e document)  ********************
% ----------------------------------------------------------------
\documentclass{article}
\usepackage[english]{babel}
\usepackage{styfiles/proof, styfiles/code, amsmath,amsthm}
\usepackage{bbold}%for typeface: mathbb
\usepackage{hyperref}
% THEOREMS -------------------------------------------------------
\newtheorem{thm}{Theorem}[section]
\newtheorem{cor}[thm]{Corollary}
\newtheorem{lem}[thm]{Lemma}
\newtheorem{prop}[thm]{Proposition}
\theoremstyle{definition}
\newtheorem{defn}[thm]{Definition}
\theoremstyle{remark}
\newtheorem{rem}[thm]{Remark}
\numberwithin{equation}{section}

% ----------------------------------------------------------------
\begin{document}

\newcommand{\env}[1]{[\![#1]\!]\kappa}
\newcommand{\round}[1]{(\!|#1|\!)}

\title{Compiler Front End}%
\author{Di Zhao}%
%\address{address}%
%\thanks{}%\sqrt{}
\date{\small{\texttt{zhaodi01@mail.ustc.edu.cn}}}%
% ----------------------------------------------------------------

\maketitle
% ----------------------------------------------------------------

Document related to the front end implementation for Monkey Compiler.

\section{Operator Priorities}

The priorities for operations are listed below:

\begin{figure}[!ht]
  % Requires \usepackage{graphicx}
  \centering
\begin{tabular}{ll}
1) & \texttt{!} \kern0.6cm \texttt{()} \kern0.6cm \texttt{int2tring} \kern0.6cm print
     \kern0.6cm \texttt{\#} \kern0.6cm Constructor application\\
2) & \texttt{$*$} \\
3) & \texttt{+} \kern0.6cm \texttt{-} \\
4) & \texttt{<} \kern0.6cm \texttt{>} \\
5) & \texttt{=}\\
6) & \texttt{andalso} \kern0.6cm \texttt{orelse} \\
7) &  Function application
\end{tabular}
  \caption{Operator Priorities (highest to lowest)}
  \label{fig-sub}
\end{figure}



\section{Other Notes}

\noindent{
\begin{description}
   \item[Case expression:]
  Default case (wild matching) is not supported. If the constructor matching fails,
  the program will simply abort?

  For a single matching branch, a variable must be provided to receive the tagged value,
  even if the corresponding constructor is just a constant value.

   \item[Numbers:]
  Integers are supported. Negative integers begin with $"\sim"$.

   \item[Variables:]
  Variable names start with any character, and consist of characters and digits.
  Underlines and "$'$" are not supported in variable names yet.

\end{description}
}


\end{document}
