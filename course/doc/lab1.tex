% ----------------------------------------------------------------
% Article Class (This is a LaTeX2e document)  ********************
% ----------------------------------------------------------------
\documentclass{article}
\usepackage[english]{babel}
\usepackage{styfiles/proof, styfiles/code, amsmath,amsthm}
\usepackage{bbold}%for typeface: mathbb
\usepackage{hyperref}
% THEOREMS -------------------------------------------------------
\newtheorem{thm}{Theorem}[section]
\newtheorem{cor}[thm]{Corollary}
\newtheorem{lem}[thm]{Lemma}
\newtheorem{prop}[thm]{Proposition}
\theoremstyle{definition}
\newtheorem{defn}[thm]{Definition}
\theoremstyle{remark}
\newtheorem{rem}[thm]{Remark}
\numberwithin{equation}{section}

% ----------------------------------------------------------------
\begin{document}

\newcommand{\env}[1]{[\![#1]\!]\kappa}
\newcommand{\round}[1]{(\!|#1|\!)}

\title{CPS Conversion}%
\author{Di Zhao}%
%\address{address}%
%\thanks{}%\sqrt{}
\date{\small{\texttt{zhaodi01@mail.ustc.edu.cn}}}%
% ----------------------------------------------------------------

\maketitle
% ----------------------------------------------------------------

This is the first assignment of the course - Advanced Topics in
Software Technologies in the 2014-2015 academic year. During this
course, we will explore the techniques in functional language
compilation. You are required to finish
the implementation of the Monkey Compiler for the simplified ML
language. Within the procedure (hopefully, if you work hard),
 you will gain deep understanding of compilation techniques
 and programming languages.

In this assignment, we will introduce the Continuation-Passing
Style (CPS), which is a program notation that makes every aspect
of control flow and data flow explicit.

First, you need to get familiar with our source language: the
simplified ML language, and then the syntax of the
continuation-passing language - our target language in lab1.
Finally, you will finish the code for CPS conversion to
translate the expressions of the source language into ones
 of the target language.

\section{ML Syntax}

Our source language is a subset of the Standard ML (SML) - a language
from the ML family. SML is a functional programming language with
 compile-time type checking and type inference, garbage collection,
 exception handling and some other features. ML is functional in that
 functions can be passed as arguments, stored in data structures, and
returned as results of function calls. Functions can be statically
nested within other functions. Computation in ML is driven by
evaluation of expressions, rather than execution of commands, as in
imperative languages such as C or Java. For more details, refer to
book "Programming in Standard ML" or \href{http://www.smlnj.org/}
{\texttt{ http://www.smlnj.org/}}.

\begin{figure}[!ht]
  % Requires \usepackage{graphicx}
  \centering
\begin{tabular}{rclr}
$e$ & $\to$ & $x$ & \emph{variable}\\
  %   &   & $t$ \textsf{primop} $t'$& \emph{arithmetic term}\\
     & $|$   & \texttt{()} & \emph{empty}\\
     & $|$   & $i$ & \emph{integer}\\
     & $|$   & $"s"$ & \emph{string}\\
     &  $|$  & \texttt{fn} $x$\texttt{ => }$e$ & \emph{abstraction}\\
     &  $|$  & $e_1\ e_2$ & \emph{application}\\
     &  $|$  & ($e_1,e_2, ..., e_n$) & \emph{tuple}\\
     &  $|$  & \texttt{\#}$i\ e$  & \emph{projection}\\
     &  $|$  & \textsf{in}$i\ e$  & \emph{tagged value}\\
     &  $|$  & \textsf{case }$e$ \textsf{of in1 }$x_1\texttt{ => } e_1|$
     \textsf{in2 }$x_2 \texttt{ => } e_2$  & \emph{case}\\
     &  $|$  & $PrimOp\ \vec{e}$ & \emph{primitive}\\
     &  $|$  & \textsf{let val }$x=e_1$\textsf{ in }$e_2$\textsf{ end}
         & \emph{let}\\
     &  $|$  & \textsf{if0 }$e$\textsf{ then }$e_1$\textsf{ else }$e_2$
         & \emph{if0}\\
     &  $|$  & \textsf{let fix }$f\ x=e_1$\textsf{ in }$e_2$\textsf{ end}
         & \emph{fix}\\
$PrimOp$ & $\to$ & $+$ & \emph{add}\\
     & $|$ & $-$ & \emph{sub}\\
     & $|$ & $*$ & \emph{times}\\
     & $|$ & $\texttt{print}$ & \emph{print}\\
     & $|$ & $\texttt{int2string}$ & \emph{toString}\\

\end{tabular}
  \caption{ML Syntax}
  \label{fig-sub}
\end{figure}

Figure 1 illustrates the syntax of our source language - a subset
 of ML. Here we use the metavariable $e$ to represent an arbitrary
 expression of the source language. Similarly, $x$ is a metavariable
ranging over variables.

To be more detailed, for example, \texttt{fn} $x$\texttt{ => }$e$
stands for the abstraction $\lambda x. e$ of the lambda-calculus, in
 which the body $e$ in turn is another expression. Then we have the application
 expression $e_1\ e_2$ which means replacing the argument in the body of
 abstraction $e_1$ with $e_2$.

($e_1,e_2, ..., e_n$) is a tuple with $n$ components. Note that in the
ML language, $n$ must be no less than 2.
 \texttt{\#}$i\ e$ represents  the projection operation that selects
  the $i$th component of $e$. Here $e$ should be a tuple. We will
not concern about type checking for now.

\textsf{in}$i\ e$ is a tagged value, that append a tag $i$ to the
expression $e$. The case expression: \textsf{case }$e$ \textsf{of in1 }
$x_1\texttt{ => } e_1|$ \textsf{in2 }$x_2 \texttt{ => } e_2$ ,
will be evaluated to the result of expression $e_1$ or $e_2$ according to
the specific tag in expression $e$, namely, here $e$ should be a tagged
value.

$PrimOp\ \vec{e}$ represents a primitive operation of expressions,
within which $\vec{e}$ stands for a list of expressions. For example,
\texttt{print} performs on a single expression, so a print expression will be:
 $\texttt{print}\ [e]$. Similarly, a sub expression will be:
 $\texttt{-}\ [e_1, e_2]$, which means $e_1 - e_2$.

\textsf{if0 }$e$\textsf{ then }$e_1$\textsf{ else }$e_2$ is a branch
expression. If $e$ evaluates to 0, then the if0 expression will be
evaluated to the evaluation result of $e_1$, else to the result of
$e_2$. $e_1$ and $e_2$ should be of the same type.

\textsf{let fix }$f\ x=e_1$\textsf{ in }$e_2$\textsf{ end} is a fix
expression that defines a recursive function $f$ with an argument
$x$ and function body $e_1$ in the expression $e_2$. $f$ is recursive
in that it can appear in $e_1$ as a free variable.

To help printing the numeric result, \texttt{int2string}
operation takes an integer and returns a corresponding
string. For example, \texttt{int2string} 4 is evaluated to "4".

"\textsf{()}" stands for the constant value of "Empty". $i$ stands for
a constant integer, $"s"$ is a constant string value.

If you are confused with the syntax, Chapter 3 and 5 of the book
\textit{Types and Programming Languages} or \href{
http://www.cs.cmu.edu/~rwh/smlbook/}{\textit{Programming in Standard ML}}
 should be helpful.\\

\noindent{
\fbox{%
\parbox{\textwidth}{%
 \textbf{Exercise 1}. Finish the code in file \texttt{ml-ast.sml} for the
function \texttt{dumpt} that
 print the ML expression to a text file. Before you start, make sure
 that you understand the datatype definition in file \texttt{ml-ast.sml}
  and file \texttt{ml-ast.sig}. Once you're done with this task, the
 text printed should be a legal function that can be executed in SML/NJ.
}
}
}



\section{CPS Syntax}

Continuation-passing style (CPS) is a program notation that makes every
aspect of control flow and data flow explicit. It also has the advantage
 that it's closely related to Church's lambda-calculus, which has a well-defined
  and well-understood meaning.

To illustrate the Continuation-passing style, consider the ML program below:
\begin{code}
fun f x =
    x + 1
\end{code}


In CPS, all the intermediate results need to be named. Also, it would be
helpful to name the return address$-$let's call it \texttt{k}. Then we can
 express the
returning of a function using \emph{continuations}. A continuation is a
function that expresses "what to do next". We can give a function a
continuation as one of its arguments, and apply the result to it when the
function returns. So the CPS version of the above program might look like this:

\begin{code}
fun f (k, x) =
    let val x\_1 = 1
    in  let val x\_2 = x + x\_1
        in k x\_2
        end
    end
\end{code}

Aside from that, the CPS-based language, i.e. our target language
in lab1 has a similar syntax to that of the ML language. Figure 2
 illustrates the syntax of the CPS language corresponding
to the ML syntax in Figure 1.

\begin{figure}[!ht]
  % Requires \usepackage{graphicx}
  \centering
\begin{tabular}{rrcl}
(terms) & $K$ & $\to$ & \textsf{letval }$x\ =\ V$ \textsf{ in } $K$ \\
        &     & $|$ & \textsf{let }$x = \pi _i\ y$\textsf{ in }$K$\\
        &     & $|$ & \textsf{letcont }$k\ x = K$\textsf{ in }$K'$\\
        &     & $|$ &  $k\ x$ \\
        &     & $|$ & $f\ k\ x$ \\
        &     & $|$ & \textsf{case} $x$ \textsf{of} $k_1\ [\!]\ k_2$\\
        &     & $|$ & \textsf{letprim} $x=PrimOp\ \vec{y}
         \texttt{ in}\ K$\\
        &     & $|$ &\textsf{if0} $x$ \textsf{then} $k_1\ \textsf{else}\ k_2$\\
        &     & $|$ &\textsf{letfix }$f\ k\ x=K$\textsf{ in }$K'$\\

(values) & $V$ & $\to$ & () \\
        &     & $|$ & $i$\\
        &     & $|$ & $"s"$\\
        &     & $|$ & ($x_1,x_2, ..., x_n$)\\
        &     & $|$ & \textsf{in}$_i\ x$\\
        &     & $|$ &  $\lambda k\ x.K$ \\

(primitives) & $PrimOp$ & $\to$ & $+$ \\
        &     & $|$ & $-$\\
        &     & $|$ & $*$\\
        &     & $|$ & \textsf{print}\\
        &     & $|$ & \textsf{int2string}\\
\end{tabular}
  \caption{CPS syntax}
  \label{fig-sub}
\end{figure}

In the CPS syntax, we introduce the metavariable $k$ to represent
a continuation. Thus there are two kinds of application:
\begin{itemize}
  \item the function application: $f\ k\ x$, in which function
$f$ takes two arguments: $k$ and $x$.
  \item the continuation application: $k\ x$, corresponding to a jump
   or a return, to pass the partial result $x$ via continuation $k$.
\end{itemize}

Besides, you may have noticed that in pair value ($x,y$), tagged value
\textsf{in}$_i\ x$, case term \textsf{case} $x$ \textsf{of} $k_1\ [\!]\ k_2$
, and the application terms, only variables rather than expressions
 (containing substructures) are allowed to exist.
Also, in term \textsf{let }$x = \pi _i\ y$\textsf{ in }$K$
(corresponding to project operation in the ML syntax),
$y$ (the subject to perform projection) is a variable.
  These alterations enforces that all intermediate results are named.

The term \texttt{Exit of string} defined in file \texttt{cps.sml}
and file \texttt{cps.sig} is considered to be a special kind of continuation
  application. It is for the purpose that initially, to translate an ML expression,
 a translate function $\kappa$ which takes a variable and return a CPS
 expression need to be given as an arguments. A reasonable choice for the initial
 $\kappa$ could be \texttt{fn x => Exit(x)}. (See the \textbf{Section 3}
 below to understand this.)\\

\noindent{
\fbox{%
\parbox{\textwidth}{%
 \textbf{Exercise 2}. Finish the code in file \texttt{cps.sml} for the
function \texttt{dump2file} that
 print the CPS expression to a text file. Before you start, make sure
 that you understand the datatype definition in file \texttt{cps.sml}
  and file \texttt{cps.sig}. Once again, the text you printed should
 be a legal function that can be executed in SML/NJ.
  You can test your implementation on the give example, or write some of your
 own examples.
}
}
}\\\\

The CPS representation is easy for optimizing compilers to manipulate and
transform. For example, we would like to perform tail-recursion elimination: If a
function f calls a function g as the very last thing it does, then instead of passing
g a return address within f, it could pass to g the return address that f was given
by f's caller. Then, when g returned, it would return directly to the caller of f.

For more details, you may refer to \href{
http://research.microsoft.com/pubs/64044/compilingwithcontinuationscontinued.pdf}
{\emph{Compiling with Continuations, Continued}}.

\section{CPS Conversion}

In this section we will discuss how to perform CPS conversion.
Expressions in ML can be translated into untyped CPS
terms using the function shown in Figure 3. This is an adaptation
of the standard higher-order one-pass call-by-value transformation
(Danvy and Filinski 1992).

 \begin{figure}[!ht]
  % Requires \usepackage{graphicx}
  \centering
\begin{align*}
 [\![\cdot]\!] \ &\ :\ \textrm{ML}\rightarrow(\textrm{Var}\rightarrow
    \textrm{CTm})\rightarrow \textrm{CTm}\\     %head
   \env{x} \  &=\  \kappa (x)\\             %var
 \env{\texttt{()}} \  &=\  \texttt{letval}\ x=()\ \texttt{in}\ \kappa (x)\\ %empty
 \env{i} \  &=\  \texttt{letval}\ x=i\ \texttt{in}\ \kappa (x)\\    %int
 \env{"s"} \  &=\  \texttt{letval}\ x="s"\ \texttt{in}\ \kappa (x)\\    %string
 \env{e_1\ e_2} \  &=\  [\![e_1]\!](\mathbb{\Lambda}z_1.[\![e_2]\!]
 (\mathbb{\Lambda}z_2.\texttt{letcont}\ k\ x=
    \kappa (x)\ \texttt{in}\ z_1\ k\ z_2))\\ %app
  \env{(e_1, ... , e_n)}
        &=\round{[e_1, ... , e_n], \texttt{nil}}(\mathbb{\Lambda}\vec{l}.\\
        &\kern1cm \texttt{letval}\ x=\texttt{tuple}(\vec{l})\
        \texttt{in}\ \kappa (x))\\            % tuple
   [\![\texttt{in}i\ e]\!]\kappa \  &=\  [\![e]\!](\mathbb{\Lambda}z.
        \texttt{letval}\ x=\texttt{in}_i\ z\ \texttt{in}\ \kappa (x))\\    %tag
  [\![\texttt{\#}i\ e]\!]\kappa \  &=\  [\![e]\!](\mathbb{\Lambda}z.
        \texttt{let}\ x=\pi_i z\ \texttt{in}\ \kappa (x))\\       %proj
  \env{\texttt{fn }x\texttt{ => }e}\  &=\ \texttt{letval}\ f=\lambda k\ x.[\![e]\!]
    (\mathbb{\Lambda}z.k\ z)\ \texttt{in}\ \kappa (f)\\     %abstract
  [\![\texttt{let\ val }x=e_1\ \texttt{in}\ e_2\ \texttt{end}]\!]\kappa
   &= \texttt{letcont}\ j\ x=[\![e_2]\!]\ \kappa\ \texttt{in}
            \ [\![e_1]\!](\mathbb{\Lambda}z.j\ z)\\     %letval
  \kern-1.2cm [\![\texttt{case}\ e\ \texttt{of\ in1}\ x_1\texttt{ => }e_1|
        \texttt{in2}\ x_2\texttt{ => } e_2]\!]\kappa
     &=\ [\![e]\!](\mathbb{\Lambda}z.\texttt{letcont}\ k_0\ x_0=
            \kappa (x_0)\ \texttt{in}\\
         &\kern1cm   \texttt{letcont}\ k_1\ x_1=
            [\![e_1]\!](\mathbb{\Lambda}z.k_0\ z)\ \texttt{in}\\
     &\kern1.2cm \texttt{letcont}\ k_2\ x_2=[\![e_2]\!]
        (\mathbb{\Lambda}z.k_0\ z)\ \texttt{in}\\
     &\kern2cm    \texttt{case}\ z\ \texttt{of}\ k_1\ [\!]\ k_2)\\    %case
     \env{PrimOp\ \vec{e}\ }
       &=\round{\vec{e}, \texttt{nil}}(\mathbb{\Lambda}\vec{l}.   \texttt{letprim}
       \ x=PrimOp\ \vec{l}\ \texttt{in}\ \kappa (x)))\\  %prims
   \env{\texttt{let fix }f\ x\texttt{ = }e_1 \texttt{ in }e_2}\  &=
   \ \texttt{letfix}\ f\ k\ x=[\![e_1]\!]
    (\mathbb{\Lambda}z.k\ z)\ \texttt{in}\ \env{e_2}\\     %fix
    \kern-1.2cm [\![\texttt{if0}\ e_1\ \texttt{then}\ e_2
        \texttt{ else}\ e_3]\!]\kappa
     &=\ [\![e_1]\!](\mathbb{\Lambda}z.\texttt{letcont}\ k_0\ x_0=
            \kappa (x_0)\ \texttt{in}\\
         &\kern1cm   \texttt{letcont}\ k_1\ x_1=
            [\![e_2]\!](\mathbb{\Lambda}z.k_0\ z)\ \texttt{in}\\
     &\kern1.2cm     \texttt{letcont}\ k_2\ x_2=[\![e_3]\!]
        (\mathbb{\Lambda}z.k_0\ z)\ \texttt{in}\\
     &\kern2cm    \texttt{if0}\ z\ \texttt{then}\ k_1\ \texttt{else}\ k_2)    %if0
\end{align*}
  \caption{CPS Conversion}
  \label{fig-sub}
\end{figure}

\begin{figure}[!ht]
  % Requires \usepackage{graphicx}
  \centering
\begin{align*}
    \round{\cdot}\ :\ \textrm{ML list * string list}&\to(\textrm{string list}\to
     \textrm{CTm})\to \textrm{CTm}\\     %head
     \round{\texttt{[]}, \omega}\eta
        &=\eta (\texttt{rev} (\omega))\\   %empty
     \round{e::es, \omega}\eta
       &=\  [\![e]\!](\mathbb{\Lambda}x.
          \round{es,x::\omega}\eta )     %some
\end{align*}

  \caption{CPS Conversion for tuples}
  \label{fig-sub}
\end{figure}

The transformation works by taking a translation-time function
$\kappa$ as argument, representing the 'context' into which the translation
of the source term is embedded. For our language, the context's
argument is a variable, as all intermediate results are named.
Note some conventions used in Figure 3: translation-time lambda abstraction
is written using $\mathbb{\Lambda}$ and translation-time application is
written $\kappa$(...), to distinguish from $\lambda$ and juxtaposition
 used to denote lambda abstraction and application in the target language.
 Also note that any object variables present in the target terms but not in
the source are assumed fresh with respect to all other bound variables.


The translate procedure $[\![\cdot]\!]$ takes two arguments:
\begin{enumerate}
\item an ML term;
\item a translation-time function that convert a variable into a CPS term.
\end{enumerate}

The translation-time functions could be generated during the translate procedure,
and will be used to generate CPS terms. As an example, consider the conversion
$[\![(\texttt{fn }x\texttt{ => }1)\ 2]\!]\kappa$, where $\kappa$ is assumed to
be $\texttt{fn }x\texttt{ => exit}\ x$:

\begin{itemize}
\item Firstly, we have rule
\begin{center}
$\env{e_1\ e_2} \  =\  [\![e_1]\!](\mathbb{\Lambda}z_1.[\![e_2]\!]
 (\mathbb{\Lambda}z_2.\texttt{letcont}\ k\ x=\kappa (x)\ \texttt{in}\ z_1\ k\ z_2))$
\end{center}
so the original expression is converted to:
\begin{center}
  $[\![\texttt{fn }x\texttt{ => }1]\!](\mathbb{\Lambda}z_1.[\![2]\!]
 (\mathbb{\Lambda}z_2.\texttt{letcont}\ k1\ x=
    \textsf{exit}\ x\ \texttt{in}\ z_1\ k1\ z_2))$.
\end{center}
$k1$ is a fresh continuation.

\item As the rule to convert an integer goes to:
\begin{center}
  $\env{i} \  =\  \texttt{letval}\ x=i\ \texttt{in}\ \kappa (x)$,
\end{center}
we have
\begin{align*}
  [\![\texttt{fn }x\texttt{ => }1]\!](\mathbb{\Lambda}z_1.&\texttt{letval}\ x1=2\\
   &\texttt{in}\ \texttt{letcont}\ k1\ x=\textsf{exit}\ x\ \texttt{in}\ z_1\ k1\ x1).
\end{align*}
Notice that the variable $x1$ is a fresh one.

\item Then, as the rule to convert an abstraction is:
\begin{center}
  $\env{\texttt{fn }x\texttt{ => }e}\ =
        \texttt{letval}\ f=\lambda k\ x.[\![e]\!]
    (\mathbb{\Lambda}z.k\ z)\ \texttt{in}\ \kappa (f)$,
\end{center}
the original expression is converted to:
\begin{align*}
  &\texttt{letval}\ f=\lambda k\ x.[\![1]\!]
    (\mathbb{\Lambda}z.k\ z)\\
     &\texttt{in}\ \texttt{letval}\ x1=2\\
   &\kern0.5cm \texttt{in}\ \texttt{letcont}\ k1\ x=
    \textsf{exit}\ x\ \texttt{in}\ f\ k1\ x1.
\end{align*}

\item Finally, apply the rule to convert a integer again, we have the result:
\begin{align*}
  &\texttt{letval}\ f=\lambda k\ x.\texttt{letval}\ x2=1\ \texttt{in}\ k\ x2\\
     &\texttt{in}\ \texttt{letval}\ x1=2\\
   &\kern0.5cm \texttt{in}\ \texttt{letcont}\ k1\ x=
    \textsf{exit}\ x\ \texttt{in}\ f\ k1\ x1.
\end{align*}
\end{itemize}

As the CPS convert for an ML tuple need to generate
 a CPS tuple with indefinite length, we need a different function.
 Here we have an additional function, denoted as $\round{\ }$ to help to
 translate an tuple expression (as shown in Figure 4). Note that
  $\round{\ }$  and $[\![\ ]\!]$.
 are mutually recursive. $\round{\ }$ takes a pair of an ML tuple and
 a string list (representing the variables that already generated) and a
 function $\eta$ that takes a string list and returns a CPS term.
The rule for a tuple that is not empty: $\round{e::es, \omega}\eta
       =[\![e]\!](\mathbb{\Lambda}x.
          \round{es,x::\omega}\eta )$,
uses [\![$e$]\!] to transfer the first expression, and then add the
corresponding variable (will be the $x$ when $(\mathbb{\Lambda}x.
          \round{es,x::\omega}\eta )$ is applied) to the list $\omega$.
When all the expressions are converted, rule $\round{\texttt{[]}, \omega}\eta
        =\eta (\texttt{rev} (\omega))$ applies the function $\eta$
        on the reversion of list $\omega$ to generate the CPS as the result.
(Function \texttt{tuple} convert a string list to a CPS tuple, function
\texttt{rev} returns the reversion of a string list.)

Think carefully about the rules above. e.g. Why in $\round{\texttt{[]}, \omega}\eta
        =\eta (\texttt{rev} (\omega))$, the list $\omega$ is converted?
It will be helpful for you to write some examples and do the translation
by hand to understand
the procedure.\\

\noindent{
\fbox{%
\parbox{\textwidth}{%
 \textbf{Exercise 3}. Finish the code in file \texttt{cps-convert.sml} for the
function \texttt{trans} that convert the ML expressions into CPS expressions.
 Before you start, make sure that you understand the rules in Figure 3 and the
 example we discussed above. The utility function \texttt{freshCont} and
 \texttt{freshVal} are provided to generate new variable and continuation names.

  You can test your implementation on the give example, or write some of your
 own examples.
}
}
}\\

\iffalse

\section{Challenge}
1. The translation rules in Figure 3 is somehow naive. For example, the conversion
of a ML case expression duplicates the context. Consider,
 for example, $f\ (\textsf{case }
x\ \textsf{of in1 } x_1 => e_1|\ \textsf{in2 } x_2 => e_2)$ whose
translation involves two calls to $f$. Modify the rule and your code to eliminate
the code redundancy it causes.

\fi
\end{document}
