% ----------------------------------------------------------------
% Article Class (This is a LaTeX2e document)  ********************
% ----------------------------------------------------------------
\documentclass{article}
\usepackage[english]{babel}
\usepackage{styfiles/proof, styfiles/code, amsmath,amsthm}
\usepackage{bbold}%for typeface: mathbb
\usepackage{hyperref}
% THEOREMS -------------------------------------------------------
\newtheorem{thm}{Theorem}[section]
\newtheorem{cor}[thm]{Corollary}
\newtheorem{lem}[thm]{Lemma}
\newtheorem{prop}[thm]{Proposition}
\theoremstyle{definition}
\newtheorem{defn}[thm]{Definition}
\theoremstyle{remark}
\newtheorem{rem}[thm]{Remark}
\numberwithin{equation}{section}

% ----------------------------------------------------------------
\begin{document}

\newcommand{\env}[1]{[\![#1]\!]\kappa}
\newcommand{\round}[1]{(\!|#1|\!)}

\title{Closure Conversion}%
\author{Di Zhao}%
%\address{address}%
%\thanks{}%\sqrt{}
\date{\small{\texttt{zhaodi01@mail.ustc.edu.cn}}}%
% ----------------------------------------------------------------

\maketitle
% ----------------------------------------------------------------

Document for the closure conversion phase.

\section{Calculating free variables}

The first step to perform closure conversion is to calculate the free variables
inside all function definitions. This allows us to decide which values need to be
 included in environments (and fetched out from the environments). Only
 after that will we be able to do closure conversion.

The source language in this phase is the continuation-passing language (Figure 1).

\begin{figure}[!ht]
  % Requires \usepackage{graphicx}
  \centering
\begin{tabular}{rrcl}
(terms) & $K$ & $\to$ & \textsf{letval }$x\ =\ V$ \textsf{ in } $K$ \\
        &     & $|$ & \textsf{letcont }$k\ x = K$\textsf{ in }$K'$\\
        &     & $|$ &  $k\ x$ \\
        &     & $|$ & $f\ k\ x$ \\
        &     & $|$ & \textsf{case} $x$ \textsf{of}
            $\overrightarrow{\texttt{in}i_j\ x_j\ \texttt{=>}\ K_j}$\\
        &     & $|$ & \textsf{letprim} $x=PrimOp\ \vec{y}
         \texttt{ in}\ K$\\
        &     & $|$ &\textsf{if} $x$ \textsf{then} $k_1\ \textsf{else}\ k_2$\\
        &     & $|$ &\textsf{letfix }$f\ k\ x=K$\textsf{ in }$K'$\\

(values) & $V$ & $\to$ & () $\ \ |$ \textsf{ true } $|$ \textsf{ false}\\
        &     & $|$ & $i\ \ $ $\ |\ $  $\ "s"$\\
        &     & $|$ & ($x_1,x_2, ..., x_n$)\\
        &     & $|$ & \textsf{in}$i\ x$\\
        &     & $|$ &  $\lambda k\ x.K$ \\
        &     & $|$ &  $\texttt{\#}i\ x$ \\

(primitive & $PrimOp$ & $\to$ & $+\ $ $|\ -\ |$ $\ *$\\
operations) &     & $|$ & $>\ |\ <\ |\ =$\\
        &     & $|$ & $\textsf{andalso }\ |\ \textsf{orelse }\ |\ \textsf{ not}$\\
        &     & $|$ & \textsf{print} $|$ \textsf{int2string}\\
\end{tabular}
  \caption{CPS syntax}
  \label{fig-sub}
\end{figure}

A free variable of an expression is one that is not bound but is used inside
the expression. Such as the $y$ and $k$ in \textsf{letval }$x=y$
\textsf{in} $k\ x$. It's worth noticing here that in \textsf{letval }$x=x$
\textsf{in} $k\ x$, the second "$x$" is also a free variable (and thus should
also be included in an environment).

Figure 2 and Figure 3 demonstrate the mutual recursive function $\mathcal{F}$ and
function $\mathcal{H}$ to calculate the free variables of a CPS term and CPS value
respectively. Both $\mathcal{F}$ and $\mathcal{H}$ return an ordered set of
 variable names. Function \texttt{set} generate a set of strings from a string list.\\

\begin{figure}[!ht]
  % Requires \usepackage{graphicx}
  \centering
\begin{align*}
\mathcal{F}\ &:\ \textrm{Cps.t} \to \textrm{string set}\\     %head
\mathcal{F}(\textsf{letval }x=V\textsf{ in }K)
    &=(\mathcal{F}(K)/x)\cup \mathcal{H}(V)\\   %letval
\mathcal{F}(\textsf{letcont }k\ x=K\textsf{ in }K')
    &=(\mathcal{F}(K)/x)\cup (\mathcal{F}(K')/k)\\   %letcont
\mathcal{F}(k\ x)
    &=\{k, x\}\\   %cont apply
\mathcal{F}(f\ k\ x)
    &=\{f, k, x\}\\   %func apply
\mathcal{F}(\textsf{case}\ x\ \textsf{of}
            \ \overrightarrow{\texttt{in}i_j\ x_j\ \texttt{=>}\ K_j})
    &=\{x\}\cup (\mathcal{F}(K_1)/x_1)\cup ... \cup (\mathcal{F}(K_n)/x_n)\\
        & (\textrm{where}\ j = 1,\ ...\ , n)\\   %case
\mathcal{F}(\textsf{letprim }x=PrimOp\ \vec{y}\texttt{ in}\ K)
    &=(\mathcal{F}(K)/x)\cup \texttt{set}(\vec{y})\\   %let prims
\mathcal{F}(\textsf{if }x\textsf{ then }k_1\ \textsf{else }k_2)
    &=\{x, k_1, k_2\}\\   %if
\mathcal{F}(\textsf{letfix }f\ k\ x=K\textsf{ in }K')
    &=(\mathcal{F}(K)-\{f,k,x\})\cup (\mathcal{F}(K')/f)   %let fix
\end{align*}
  \caption{Calculating Free Variables in CPS terms}
  \label{fig-sub}
\end{figure}

\begin{figure}[!ht]
  % Requires \usepackage{graphicx}
  \centering
\begin{align*}
\mathcal{H}\ &:\ \textrm{Cps.v} \to \textrm{string set}\\     %head
\mathcal{H}(()) &={\O}\\   %empty
\mathcal{H}(\texttt{true}) &={\O}\\   %true
\mathcal{H}(\texttt{false}) &={\O}\\   %false
\mathcal{H}(i) &={\O}\\   %integer
\mathcal{H}("s") &={\O}\\   %string
\mathcal{H}((x_1, x_2,\ ...\ , x_n)) &=\{x_1, x_2,\ ...\ , x_n\}\\   %tuple
\mathcal{H}(\textsf{in}i\ x) &= \{x\}\\        %tag
\mathcal{H}(\texttt{\#}i\ x) &= \{x\}\\        %proj
\mathcal{H}(\lambda k\ x.K) &= \mathcal{F}(K)-\{k, x\}      %abstraction
\end{align*}
  \caption{Calculating Free Variables in CPS values}
  \label{fig-sub}
\end{figure}


\section{Closure Syntax}

The target language is a closure-passing language. Wherever
a function is needed to be returned as a result, or passed as an argument,
a corresponding closure will be transmitted instead. Aside from that, the
Closure syntax (shown in Figure 4) is similar to the CPS syntax.


\begin{figure}[!ht]
  % Requires \usepackage{graphicx}
  \centering
\begin{tabular}{rrcl}
(terms) & $K$ & $\to$ & \textsf{letval }$x\ =\ V$ \textsf{ in } $K$ \\
        &     & $|$ & \textsf{let }$x = \texttt{\#}i\ y$\textsf{ in }$K$\\
        &     & $|$ & \textsf{letcont }$k\ env\ x = K$\textsf{ in }$K'$\\
        &     & $|$ &  $k\ env\ x$ \\
        &     & $|$ & $f\ env\ k\ x$ \\
        &     & $|$ & \textsf{case} $x$ \textsf{of}
             $\overrightarrow{\textsf{in}i_j\ x_j \Rightarrow K_j}$\\
        &     & $|$ & \textsf{letprim} $x=PrimOp\ \vec{y}
         \textsf{ in}\ K$\\
        &     & $|$ &\textsf{if} $x$ \textsf{then} $K\ \textsf{else}\ K'$\\
        &     & $|$ &\textsf{letfix }$f\ env\ k\ x=K$\textsf{ in }$K'$\\

(values) & $V$ & $\to$ & () \\
        &     & $|$ & \texttt{true}\\
        &     & $|$ & \texttt{false}\\
        &     & $|$ & $i$\\
        &     & $|$ & $"s"$\\
        &     & $|$ & ($x_1,x_2,\ ...\ , x_n$)\\
        &     & $|$ & \texttt{in}$i\ x$\\
        &     & $|$ &  $\lambda env\ k\ x.K$ \\

(primitive & $PrimOp$ & $\to$ & $+\ $ $|\ -\ |$ $\ *$\\
operations) &     & $|$ & $>\ |\ <\ |\ =$\\
        &     & $|$ & $\textsf{andalso }\ |\ \textsf{orelse }\ |\ \textsf{ not}$\\
        &     & $|$ & \textsf{print} $|$ \textsf{int2string}\\
\end{tabular}
  \caption{Closure syntax}
  \label{fig-sub}
\end{figure}


\section{Closure Conversion}


\begin{figure}[!ht]
  % Requires \usepackage{graphicx}
  \centering
\begin{align*}
\Theta\ &:\ \textrm{Cps.t} \to \textrm{Closure.t}\\     %head
\Theta(\textsf{let val }x=\texttt{\#}i\ y\textsf{ in }K)
    &=\textsf{let }x=\texttt{\#}i\ y\textsf{ in }\Theta(K)\\   %let
\Theta(\textsf{letcont }k\ x=K\textsf{ in }K')
    &=\textsf{letcont }k_{code}\ env\ x=\\
    &\kern1cm   \texttt{let }y_1=\texttt{\#}1\ env\ \texttt{in}\\
    &\kern1.4cm   \texttt{let }y_2=\texttt{\#}2\ env\ \texttt{in}\\
    &\kern1.4cm   ...\\
    &\kern1.8cm   \texttt{let }y_m=\texttt{\#}m\ env\ \texttt{in}\ \Theta(K)\\
    &\kern0.6cm\textsf{in letval }env'=(y_1, y_2, \ ...\ ,y_m)\texttt{ in}\\
    &\kern1.6cm\textsf{letval }k=(k_{code}, env')\texttt{ in }\Theta(K')\\ %letcont
    (&where\ \{y_1,\ ...\ ,y_m\}=\mathcal{F}(K)-\{x\})\\
\Theta(k\ x)
    &=\texttt{let }k_{code}=\texttt{\#}1\ k\ \texttt{in}\\
    &\kern1cm   \texttt{let }env=\texttt{\#}2\ k\ \texttt{in}\\
    &\kern1.5cm  k_{code}\ env\ x\\   %cont apply
\Theta(f\ k\ x)
    &=\texttt{let }f_{code}=\texttt{\#}1\ f\ \texttt{in}\\
    &\kern1cm   \texttt{let }env=\texttt{\#}2\ f\ \texttt{in}\\
    &\kern1.5cm  f_{code}\ env\ k \ x\\   %func apply
\Theta(\textsf{case}\ x\ \textsf{of}
           \ \overrightarrow{\texttt{in}i_j\ x_j\ \texttt{=>}\ K_j})
    &=\textsf{case }x\ \textsf{of}\ \overrightarrow{\textsf{in}i_j\ x_j \Rightarrow
     \Theta(K_j)}\\   %case
\Theta(\textsf{letprim }x=PrimOp\ \vec{y}\texttt{ in}\ K)
    &=\textsf{letprim }x=PrimOp\ \vec{y}\texttt{ in}\ \Theta(K)\\   %let prims
\Theta(\textsf{if }x\textsf{ then }k_1\ \textsf{else }k_2)
    &=\textsf{letval }x_1=()\textsf{ in}\\
    &\kern1cm \textsf{if }x\textsf{ then }\Theta(k_1\ x_1)
    \ \textsf{else }\Theta(k_2\ x_1)\\
\Theta(\textsf{letfix }f\ k\ x=K\textsf{ in }K')   %if
    &=\textsf{letfix }f_{code}\ env\ k\ x=\\
    &\kern1cm   \texttt{letval }f=(f_{code}, env)\ \texttt{in}\\
    &\kern1.4cm   \texttt{let }y_1=\texttt{\#}1\ env\ \texttt{in}\\
    &\kern1.8cm   \texttt{let }y_2=\texttt{\#}2\ env\ \texttt{in}\\
    &\kern1.8cm   ...\\
    &\kern2.2cm   \texttt{let }y_m=\texttt{\#}m\ env\ \texttt{in}\ \Theta(K)\\
    &\kern0.6cm\textsf{in letval }env'=(y_1, y_2, \ ...\ ,y_m)\texttt{ in}\\
    &\kern1cm\textsf{letval }f=(f_{code}, env')\texttt{ in }\Theta(K')\\  %let fix
    (&where\ \{y_1,\ ...\ ,y_m\}=\mathcal{F}(K)-\{f,k,x\})
\end{align*}
  \caption{Closure Conversion for CPS terms}
  \label{fig-sub}
\end{figure}

\begin{figure}[!ht]
  % Requires \usepackage{graphicx}
  \centering
\begin{align*}
\Theta\ &:\ \textrm{Cps.t} \to \textrm{Closure.t}\\     %head
\Theta(\textsf{letval }x=()\textsf{ in }K)
    &=\textsf{letval }x=()\textsf{ in }\Theta(K)\\   %letval: empty
\Theta(\textsf{letval }x=\texttt{true}\textsf{ in }K)
    &=\textsf{letval }x=\texttt{true}\textsf{ in }\Theta(K)\\   %letval: true
\Theta(\textsf{letval }x=\texttt{false}\textsf{ in }K)
    &=\textsf{letval }x=\texttt{false}\textsf{ in }\Theta(K)\\   %letval: false
\Theta(\textsf{letval }x=i\textsf{ in }K)
    &=\textsf{letval }x=i\textsf{ in }\Theta(K)\\   %letval: int
\Theta(\textsf{letval }x="s"\textsf{ in }K)
    &=\textsf{letval }x="s"\textsf{ in }\Theta(K)\\   %letval: string
\Theta(\textsf{letval }x=(x_1, x_2,\ ...\ , x_n)\textsf{ in }K)
    &=\textsf{letval }x=(x_1, x_2,\ ...\ , x_n)\textsf{ in }
    \Theta(K)\\ %letval:tuple
\Theta(\textsf{letval }x=\textsf{in}_i\ y\textsf{ in }K)
    &=\textsf{letval }x=\textsf{in}_i\ y\textsf{ in }\Theta(K)\\%letval: tag
\Theta(\textsf{letval }x=\lambda k\ z.K\textsf{ in }K')
    &=\textsf{letval }x_{code}=\lambda env\ k\ z.\\
    &\kern1cm   \texttt{let }y_1=\texttt{\#}1\ env\ \texttt{in}\\
    &\kern1.4cm   \texttt{let }y_2=\texttt{\#}2\ env\ \texttt{in}\\
    &\kern1.4cm   ...\\
    &\kern1.8cm   \texttt{let }y_m=\texttt{\#}m\ env\ \texttt{in}\ \Theta(K)\\
    &\kern0.6cm\textsf{in letval }env'=(y_1, y_2, \ ...\ ,y_m)\texttt{ in}\\
    &\kern1.6cm\textsf{letval }x=(x_{code}, env')\texttt{ in }\Theta(K')\\
    (&where\ \{y_1,\ ...\ ,y_m\}=\mathcal{F}(K)-\{k, z\})%letval: func
\end{align*}
  \caption{Closure Conversion for CPS terms (continued)}
  \label{fig-sub}
\end{figure}



\noindent{
Updates:
\begin{description}
  \item[15-7-3:]
  Change the form of \texttt{case} in \texttt{closure.sig}, \texttt{closure.sml},
  and changed the respective converting rule in \texttt{closure-convert.sml}
  to enable multiple cases .
  
  \item[15-9-5:]
  Added boolean values and corresponding operations. Changed \texttt{if0}
  expression into \texttt{if} expression.
\end{description}
}

\end{document}
