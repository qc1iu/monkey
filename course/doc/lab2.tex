% ----------------------------------------------------------------
% Article Class (This is a LaTeX2e document)  ********************
% ----------------------------------------------------------------
\documentclass{article}
\usepackage[english]{babel}
\usepackage{styfiles/proof, styfiles/code, amsmath,amsthm}
\usepackage{bbold}%for typeface: mathbb
\usepackage{hyperref}
% THEOREMS -------------------------------------------------------
\newtheorem{thm}{Theorem}[section]
\newtheorem{cor}[thm]{Corollary}
\newtheorem{lem}[thm]{Lemma}
\newtheorem{prop}[thm]{Proposition}
\theoremstyle{definition}
\newtheorem{defn}[thm]{Definition}
\theoremstyle{remark}
\newtheorem{rem}[thm]{Remark}
\numberwithin{equation}{section}

% ----------------------------------------------------------------
\begin{document}

\newcommand{\env}[1]{[\![#1]\!]\kappa}
\newcommand{\round}[1]{(\!|#1|\!)}

\title{Closure Conversion}%
\author{Di Zhao}%
%\address{address}%
%\thanks{}%\sqrt{}
\date{\small{\texttt{zhaodi01@mail.ustc.edu.cn}}}%
% ----------------------------------------------------------------

\maketitle
% ----------------------------------------------------------------

This is the second assignment of Advanced Topics in Software
Technologies. Previously in this course we have learned about the concept
 of the Continuation-Passing Style (CPS), and implemented the CPS convert function
 from the ML source language to the continuation-passing language. In this
 assignment, we will take a step further by looking at another crucial
 phase in compiling functional languages - closure conversion.

To start with, we will introduce an example to help understanding the concept
 and purpose of closure conversion. Then you'll need to implement the preparation
 step for closure conversion - free variables calculating. After that, you need
 to get familiar with the syntax of the target language in this lab - a
 closure-passing language. Finally, you will finish the code for
 closure conversion.

After this assignment, you will gain deeper understanding about functional
 programming languages.

\section{What \& Why}

In our source language, function definitions have nested
static scope, a function can refer to variables of the function in
 which it is defined. However the notion of a function as a machine-code
address does not provide for free variables. To elaborate the problem and
its solution, let us consider how to encode this feature in C programming
language.

Below is a fragment of code that defines a function \texttt{f} inside another
function \texttt{g}, while \texttt{f} refers to the argument \texttt{x} of
function \texttt{g} in its return statement. Namely, there is a free existence
of variable \texttt{x} in \texttt{f}. Without any doubt, C language
convention will not allow that. Moreover, to use function \texttt{f} later, we want
the variable \texttt{x} to be accessible even after the function \texttt{g}
has returned.

\begin{code}
int (*g (int x))(int) \{
  int f (int y)\{
    return (x+y) ;
  \}
  return f;
\}
\end{code}

Supposing that the above code was legal, to calculate 3 + 4 by using the above
function, the main function may look like this:

\begin{code}
int main ()\{
  int (*h)(int) = g (3);
  printf ("%d", h (4));
  return 0;
\}
\end{code}


To solve the problems, we use an \emph{environment} to pass
a function the free variables it needs (such as \texttt{x} here). And
a function will be converted to a \emph{closure} which contains function code
and the environment.

Therefore the above code is converted to
\footnote{To compile this code with a C compiler, one more step is needed:
function \texttt{f} has to be lifted to top level. We will look at it
in the next lab.}:

\begin{code}
void **g (int *env, int x)\{
  int f (int *env, int y)\{
    int x = env[0];
    return (x+y) ;
  \}
  int *env0 = malloc(sizeof(int)*1);
  env0[0] = x;      //build the environment for f
  void **tuple = malloc(sizeof(void *)*2);
  tuple[0] = f;
  tuple[1] = env0;  //construct the closure
  return tuple;     //return a closure instead of a function
\}
\end{code}

Functions now take an extra argument \texttt{env} as the environment, which
contains the original free variables (such as variable \texttt{x} in \texttt{f} here).
\texttt{g} will now return a closure instead of a function, the closure contains
code for \texttt{f} and the environment of \texttt{f}. To use the above function
to calculate 3 + 4,
we may run:

\begin{code}
void main ()\{
  int * env1 = 0;
  void ** closure = g(env1, 3);  //g returns the closure of f.
  void * code = closure[0];   //get function code from the closure
  int env = closure[1];       //get environment from the closure
  printf("%d", ((void * (*)())code)(env, 4));
  //apply the code to the environment and original arguments
\}
\end{code}

To perform a function call, we need to unfold the
corresponding closure and apply the function code (first component of the
closure) to the environment (second component of the closure) and its arguments.

To be mentioned, as function \texttt{f} can be returned as the result of \texttt{g},
or stored into a data structure and invoked after \texttt{g} returns,
the variables in the activation record of \texttt{g}
may now be used after \texttt{g} has returned. This means that activation records
can no longer be stored on a stack, but must instead be allocated on a heap.
We will discuss about this in more detail later in Garbage Collection.

Make sure that you have fully understand the above example before you continue.\\

\section{Calculating free variables}

The first step to perform closure conversion is to calculate the free variables
inside all function definitions. This allows us to decide which values need to be
 included in environments (and fetched out from the environments). Only
 after that will we be able to do closure conversion.

The source language in this lab is exactly the target language in the previous one,
 i.e., the continuation-passing language. You can review its syntax with Figure 1.

\begin{figure}[!ht]
  % Requires \usepackage{graphicx}
  \centering
\begin{tabular}{rrcl}
(terms) & $K$ & $\to$ & \textsf{letval }$x\ =\ V$ \textsf{ in } $K$ \\
        &     & $|$ & \textsf{let }$x = \pi _i\ y$\textsf{ in }$K$\\
        &     & $|$ & \textsf{letcont }$k\ x = K$\textsf{ in }$K'$\\
        &     & $|$ &  $k\ x$ \\
        &     & $|$ & $f\ k\ x$ \\
        &     & $|$ & \textsf{case} $x$ \textsf{of} $k_1\ [\!]\ k_2$\\
        &     & $|$ & \textsf{letprim} $x=PrimOp\ \vec{y}
         \texttt{ in}\ K$\\
        &     & $|$ &\textsf{if0} $x$ \textsf{then} $k_1\ \textsf{else}\ k_2$\\
        &     & $|$ &\textsf{letfix }$f\ k\ x=K$\textsf{ in }$K'$\\

(values) & $V$ & $\to$ & () \\
        &     & $|$ & $i$\\
        &     & $|$ & $"s"$\\
        &     & $|$ & ($x_1,x_2,\ ...\ , x_n$)\\
        &     & $|$ & \textsf{in}$_i\ x$\\
        &     & $|$ &  $\lambda k\ x.K$ \\

(primitives) & $PrimOp$ & $\to$ & $+$ \\
        &     & $|$ & $-$\\
        &     & $|$ & $*$\\
        &     & $|$ & \textsf{print}\\
        &     & $|$ & \textsf{int2string}\\
\end{tabular}
  \caption{CPS syntax}
  \label{fig-sub}
\end{figure}

A free variable of an expression is one that is not bound but is used inside
the expression. Such as the $y$ and $k$ in \textsf{letval }$x=y$
\textsf{in} $k\ x$. It's worth noticing here that in \textsf{letval }$x=x$
\textsf{in} $k\ x$, the second "$x$" is also a free variable (and thus should
also be included in an environment).

Figure 2 and Figure 3 demonstrate the mutual recursive function $\mathcal{F}$ and
function $\mathcal{H}$ to calculate the free variables of a CPS term and CPS value
respectively. Both $\mathcal{F}$ and $\mathcal{H}$ return an ordered set of
 variable names. Function \texttt{set} generate a set of strings from a string list.\\

\begin{figure}[!ht]
  % Requires \usepackage{graphicx}
  \centering
\begin{align*}
\mathcal{F}\ &:\ \textrm{Cps.t} \to \textrm{string set}\\     %head
\mathcal{F}(\textsf{letval }x=V\textsf{ in }K)
    &=(\mathcal{F}(K)/x)\cup \mathcal{H}(V)\\   %letval
\mathcal{F}(\textsf{let }x=\pi _i\ y\textsf{ in }K)
    &=(\mathcal{F}(K)/x)\cup \{y\}\\   %let
\mathcal{F}(\textsf{letcont }k\ x=K\textsf{ in }K')
    &=(\mathcal{F}(K)/x)\cup (\mathcal{F}(K')/k)\\   %letcont
\mathcal{F}(k\ x)
    &=\{k, x\}\\   %cont apply
\mathcal{F}(f\ k\ x)
    &=\{f, k, x\}\\   %func apply
\mathcal{F}(\textsf{case }x\textsf{ of }k_1\ [\!]\ k_2)
    &=\{x, k_1, k_2\}\\   %case
\mathcal{F}(\textsf{letprim }x=PrimOp\ \vec{y}\texttt{ in}\ K)
    &=(\mathcal{F}(K)/x)\cup \texttt{set}(\vec{y})\\   %let prims
\mathcal{F}(\textsf{if0 }x\textsf{ then }k_1\ \textsf{else }k_2)
    &=\{x, k_1, k_2\}\\   %if0
\mathcal{F}(\textsf{letfix }f\ k\ x=K\textsf{ in }K')
    &=(\mathcal{F}(K)-\{f,k,x\})\cup (\mathcal{F}(K')/f)   %let fix
\end{align*}
  \caption{Calculating Free Variables in CPS terms}
  \label{fig-sub}
\end{figure}

\begin{figure}[!ht]
  % Requires \usepackage{graphicx}
  \centering
\begin{align*}
\mathcal{H}\ &:\ \textrm{Cps.v} \to \textrm{string set}\\     %head
\mathcal{H}(()) &={\O}\\   %empty
\mathcal{H}(i) &={\O}\\   %integer
\mathcal{H}("s") &={\O}\\   %string
\mathcal{H}((x_1, x_2,\ ...\ , x_n)) &=\{x_1, x_2,\ ...\ , x_n\}\\   %tuple
\mathcal{H}(\textsf{in}_i\ x) &= \{x\}\\        %tag
\mathcal{H}(\lambda k\ x.K) &= \mathcal{F}(K)-\{k, x\}      %abstraction
\end{align*}
  \caption{Calculating Free Variables in CPS values}
  \label{fig-sub}
\end{figure}

\noindent{
\fbox{%
\parbox{\textwidth}{%
 \textbf{Exercise 1}. Read the formulas above carefully. Make sure you
 understand each one before moving on to the next section.

 $\ \ \ $For example, consider this question: In the first rule:
 \begin{center}
 $\mathcal{F}(\textsf{letval }x=V\textsf{ in }K)
    =(\mathcal{F}(K)/x)\cup \mathcal{H}(V)$,
 \end{center}
 can we remove $x$ after the union of the two sets? Namely, will
\begin{center}
 $\mathcal{F}(\textsf{letval }x=V\textsf{ in }K)
    =(\mathcal{F}(K)\cup \mathcal{H}(V))/x$
\end{center}
 be correct?
}
}
}

\section{Closure Syntax}

In this lab, the target language is a closure-passing language. Wherever
a function is needed to be returned as a result, or passed as an argument,
a corresponding closure will be transmitted instead. Aside from that, the
Closure syntax (shown in Figure 4) is similar to the CPS syntax.


\begin{figure}[!ht]
  % Requires \usepackage{graphicx}
  \centering
\begin{tabular}{rrcl}
(terms) & $K$ & $\to$ & \textsf{letval }$x\ =\ V$ \textsf{ in } $K$ \\
        &     & $|$ & \textsf{let }$x = \pi _i\ y$\textsf{ in }$K$\\
        &     & $|$ & \textsf{letcont }$k\ env\ x = K$\textsf{ in }$K'$\\
        &     & $|$ &  $k\ env\ x$ \\
        &     & $|$ & $f\ env\ k\ x$ \\
        &     & $|$ & \textsf{case} $x$ \textsf{of in}$_1\ x_1 \Rightarrow K
                    \ |$ \textsf{in}$_2\ x_2 \Rightarrow K'$\\
        &     & $|$ & \textsf{letprim} $x=PrimOp\ \vec{y}
         \texttt{ in}\ K$\\
        &     & $|$ &\textsf{if0} $x$ \textsf{then} $K\ \textsf{else}\ K'$\\
        &     & $|$ &\textsf{letfix }$f\ env\ k\ x=K$\textsf{ in }$K'$\\

(values) & $V$ & $\to$ & () \\
        &     & $|$ & $i$\\
        &     & $|$ & $"s"$\\
        &     & $|$ & ($x_1,x_2,\ ...\ , x_n$)\\
        &     & $|$ & \textsf{in}$_i\ x$\\
        &     & $|$ &  $\lambda env\ k\ x.K$ \\

(primitives) & $PrimOp$ & $\to$ & $+$ \\
        &     & $|$ & $-$\\
        &     & $|$ & $*$\\
        &     & $|$ & \textsf{print}\\
        &     & $|$ & \textsf{int2string}\\
\end{tabular}
  \caption{Closure syntax}
  \label{fig-sub}
\end{figure}

Compared with the CPS syntax, there are basically two aspects of modifications:
\begin{itemize}
  \item For function definitions, an extra formal parameter $env$ is attached
    as the environment of a function:
    \begin{itemize}
      \item \textsf{letcont }$k\ \underline{env}\ x = K$\textsf{ in }$K'$
      \item \textsf{letfix }$f\ \underline{env}\ k\ x=K$\textsf{ in }$K'$
      \item  $\lambda \underline{env}\ k\ x.K$
    \end{itemize}
  \item For function applications (continuations are treated as functions after
  the CPS phase), an extra actual parameter $env$ is provided:
    \begin{itemize}
      \item $k\ \underline{env}\ x$
      \item $f\ \underline{env}\ k\ x$
    \end{itemize}
\end{itemize}

\noindent{
\fbox{%
\parbox{\textwidth}{%
 \textbf{Exercise 2}. Finish the function \texttt{freeVarTerm} and \texttt{freeVarValue}
  in file \texttt{closure-convert.sml}. Function \texttt{freeVarTerm} calculate
  the free variables in a CPS term, and function \texttt{freeVarValue} calculate
  the free variables in a CPS value.


  $\ \ $ To rule out redundant calculation and for convenience sake, we store the
  free variables in closure syntax tree directly. Therefore during
  the calculation, the CPS syntax term is converted into a closure passing syntax
  tree carrying information about free variables. This translation is just a naive
  substitution without doing the actual closure conversion, thus will not
  carry correctness. We will do the actual closure conversion in the next
  section.

  $\ \ $ To encode this, function \texttt{freeVarTerm} and \texttt{freeVarValue}
  return a pair of the converted syntax item and the free variable set of the item,
  both of them are constructed recursively.

  $\ \ $ There are 3 kind of syntax items where a function will be defined:
  \begin{quote}
  \textsf{letcont }$k\ env\ x = K$\textsf{ in }$K'$\\
  \textsf{letfix }$f\ env\ k\ x=K$\textsf{ in }$K'$\\
   $\lambda env\ k\ x.K$.
  \end{quote}
  Information about free variables of a function is stored where it is defined.
  You will find an extra component - a string list is attached in closure syntax
  definitions for the three items in our code.

  $\ \ $ Pay special attention that the free variable sets these three syntax items
  carrying only contain the free variables of \emph{the function definition}
  (i.e., free variables of the function body with the arguments excluded), rather
  than of the entire syntax item. You'll understand this better in the next
  section.

  $\ \ $ The utility function \texttt{freshCont}, \texttt{freshEnv} and
  \texttt{freshVal} are provided to generate new exclusive names.

  $\ \ $ You can test your implementation with some examples and examine the
  printed free variable sets to check whether you
  have implemented the functions properly.
}}}\\\\

\noindent{
\fbox{%
\parbox{\textwidth}{%
 \textbf{Exercise 3}. Finish the code in file \texttt{closure.sml} to pretty
 print a \texttt{Closure.t}. However as we haven't dealt with types, the output
 is not likely to pass the compilation of SML/NJ.

  $\ \ $ You should print out the information of free variables stored in the closure
  syntax tree to check your \texttt{freeVarTerm} and \texttt{freeVarValue}
  function with some examples before continue.
}
}
}\\\\

\section{Closure Conversion}

In this section we will discuss how to perform closure conversion.
The function $\Theta$ described in Figure 5 and Figure 6 translate a CPS term
into a closure passing term.\\

While some CPS terms are translated easily with a recursive application of
$\Theta$, some other CPS terms call for special concern. The latter can be
divided into a few categories:

\begin{enumerate}
  \item Function definitions: \textsf{letcont }$k\ x = K$\textsf{ in }$K'$,
  \textsf{letfix }$f\ k\ x=K$\textsf{ in }$K'$, and $\lambda k\ x.K$.

  Recalling what we have discussed in Section 1, there are some alterations for
  a syntax item where a function is defined:
  \begin{enumerate}
    \item A formal parameter representing the environment is inserted (the $env$
    field of the target syntax item.).
    \footnote{However you may want to finish this job during the free variable
    calculating phase, as the CPS terms are already converted to a closure syntax
    term there.}
    \item In a function body, all free variables must be fetched out from the
    environment first. This is done by adding a series of nested \texttt{let}
    expression. After the previous section, the free variables we need are
    already stored in the syntax tree.
    \item Before entering the "\texttt{in}" body (say $K'$ of
        \textsf{letcont }$k\ env\ x = K$\textsf{ in }$K'$\\),
    an actual environment should be generated, by inserting a \texttt{letval}
    expression that binds a variable with a tuple of the function's free
    variables. Because we are out of the function definition here, these variables
    are not "free" now, which means we can access them directly.

    $\ \ $ Then the closure containing the function name and the environment variable
    is generated via another \texttt{letval} clause. Note that
    \emph{the closure should have the same name with the original function},$\ \ \ \ $
    which allows function applications in the "in" body to find the right closure
    for the right function.
    \item The term $\textsf{letfix }f\ k\ x=K\textsf{ in }K'$
    requires special treatment in that the function $f$ might be used inside its
    function body $K$. The problem is solved easily by inserting
    "$\texttt{letval }f=(f_{code}, env)\ \texttt{in}\ ...$" outside $K$, to define
    a closure with the name $f$ consisted of the (new) function name $f_{code}$ and
    the parameter $env$.
  \end{enumerate}
  \item Function applications: $k\ x$, $f\ k\ x$.


  To perform a function application, we need to fetch the function code and the
  environment from the corresponding closure (of the same name with the orignal
  function in CPS term), and apply the code to the environment and the original
  arguments.


  To unfold the closure, two \texttt{let} expressions are inserted.
\end{enumerate}



\begin{figure}[!ht]
  % Requires \usepackage{graphicx}
  \centering
\begin{align*}
\Theta\ &:\ \textrm{Cps.t} \to \textrm{Closure.t}\\     %head
\Theta(\textsf{let }x=\pi _i\ y\textsf{ in }K)
    &=\textsf{let }x=\pi _i\ y\textsf{ in }\Theta(K)\\   %let
\Theta(\textsf{letcont }k\ x=K\textsf{ in }K')
    &=\textsf{letcont }k_{code}\ env\ x=\\
    &\kern1cm   \texttt{let }y_1=\pi _1\ env\ \texttt{in}\\
    &\kern1.4cm   \texttt{let }y_2=\pi _2\ env\ \texttt{in}\\
    &\kern1.4cm   ...\\
    &\kern1.8cm   \texttt{let }y_m=\pi _m\ env\ \texttt{in}\ \Theta(K)\\
    &\kern0.6cm\textsf{in letval }env'=(y_1, y_2, \ ...\ ,y_m)\texttt{ in}\\
    &\kern1.6cm\textsf{letval }k=(k_{code}, env')\texttt{ in }\Theta(K')\\ %letcont
    (&where\ \{y_1,\ ...\ ,y_m\}=\mathcal{F}(K)-\{x\})\\
\Theta(k\ x)
    &=\texttt{let }k_{code}=\pi _1\ k\ \texttt{in}\\
    &\kern1cm   \texttt{let }env=\pi _2\ k\ \texttt{in}\\
    &\kern1.5cm  k_{code}\ env\ x\\   %cont apply
\Theta(f\ k\ x)
    &=\texttt{let }f_{code}=\pi _1\ f\ \texttt{in}\\
    &\kern1cm   \texttt{let }env=\pi _2\ f\ \texttt{in}\\
    &\kern1.5cm  f_{code}\ env\ k \ x\\   %func apply
\Theta(\textsf{case }x\textsf{ of }k_1\ [\!]\ k_2)
    &=\textsf{case }x\textsf{ of in}_1\ x_1\Rightarrow \Theta(k_1\ x_1)\\
    &\kern1.6cm  |\ \textsf{in}_2\ x_2\Rightarrow \Theta(k_2\ x_2)\\   %case
\Theta(\textsf{letprim }x=PrimOp\ \vec{y}\texttt{ in}\ K)
    &=\textsf{letprim }x=PrimOp\ \vec{y}\texttt{ in}\ \Theta(K)\\   %let prims
\Theta(\textsf{if0 }x\textsf{ then }k_1\ \textsf{else }k_2)
    &=\textsf{letval }x_1=()\textsf{ in}\\
    &\kern1cm \textsf{if0 }x\textsf{ then }\Theta(k_1\ x_1)
    \ \textsf{else }\Theta(k_2\ x_1)\\
\Theta(\textsf{letfix }f\ k\ x=K\textsf{ in }K')   %if0
    &=\textsf{letfix }f_{code}\ env\ k\ x=\\
    &\kern1cm   \texttt{letval }f=(f_{code}, env)\ \texttt{in}\\
    &\kern1.4cm   \texttt{let }y_1=\pi _1\ env\ \texttt{in}\\
    &\kern1.8cm   \texttt{let }y_2=\pi _2\ env\ \texttt{in}\\
    &\kern1.8cm   ...\\
    &\kern2.2cm   \texttt{let }y_m=\pi _m\ env\ \texttt{in}\ \Theta(K)\\
    &\kern0.6cm\textsf{in letval }env'=(y_1, y_2, \ ...\ ,y_m)\texttt{ in}\\
    &\kern1cm\textsf{letval }f=(f_{code}, env')\texttt{ in }\Theta(K')\\  %let fix
    (&where\ \{y_1,\ ...\ ,y_m\}=\mathcal{F}(K)-\{f,k,x\})
\end{align*}
  \caption{Closure Conversion for CPS terms}
  \label{fig-sub}
\end{figure}

\begin{figure}[!ht]
  % Requires \usepackage{graphicx}
  \centering
\begin{align*}
\Theta\ &:\ \textrm{Cps.t} \to \textrm{Closure.t}\\     %head
\Theta(\textsf{letval }x=()\textsf{ in }K)
    &=\textsf{letval }x=()\textsf{ in }\Theta(K)\\   %letval: empty
\Theta(\textsf{letval }x=i\textsf{ in }K)
    &=\textsf{letval }x=i\textsf{ in }\Theta(K)\\   %letval: int
\Theta(\textsf{letval }x="s"\textsf{ in }K)
    &=\textsf{letval }x="s"\textsf{ in }\Theta(K)\\   %letval: string
\Theta(\textsf{letval }x=(x_1, x_2,\ ...\ , x_n)\textsf{ in }K)
    &=\textsf{letval }x=(x_1, x_2,\ ...\ , x_n)\textsf{ in }
    \Theta(K)\\ %letval:tuple
\Theta(\textsf{letval }x=\textsf{in}_i\ y\textsf{ in }K)
    &=\textsf{letval }x=\textsf{in}_i\ y\textsf{ in }\Theta(K)\\%letval: tag
\Theta(\textsf{letval }x=\lambda k\ z.K\textsf{ in }K')
    &=\textsf{letval }x_{code}=\lambda env\ k\ z.\\
    &\kern1cm   \texttt{let }y_1=\pi _1\ env\ \texttt{in}\\
    &\kern1.4cm   \texttt{let }y_2=\pi _2\ env\ \texttt{in}\\
    &\kern1.4cm   ...\\
    &\kern1.8cm   \texttt{let }y_m=\pi _m\ env\ \texttt{in}\ \Theta(K)\\
    &\kern0.6cm\textsf{in letval }env'=(y_1, y_2, \ ...\ ,y_m)\texttt{ in}\\
    &\kern1.6cm\textsf{letval }x=(x_{code}, env')\texttt{ in }\Theta(K')\\
    (&where\ \{y_1,\ ...\ ,y_m\}=\mathcal{F}(K)-\{k, z\})%letval: func
\end{align*}
  \caption{Closure Conversion for CPS terms (continued)}
  \label{fig-sub}
\end{figure}



\noindent{
\fbox{%
\parbox{\textwidth}{%
 \textbf{Exercise 4}. Finish the code in file \texttt{closure-convert.sml} for the
function \texttt{convert} that performs the actual closure conversion.
 Before you start, make sure that you understand the rules in Figure 5 and 6
 thoroughly.

 $\ \ $ You can test your implementation on the give example, or write some of your
 own examples.
}
}
}\\

\end{document}
